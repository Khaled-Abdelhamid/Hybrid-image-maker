%%%%%%%%%%%%%%%%%%%%%%%%%%%%%%%%%%%%%%%%%%%%%%%%%%%%%%%%%%%%%%%%%%%%%%%%%%%%%%%%%%%%%%%%%%%%%%%%
%
% CSCI 1430 Written Question Template
%
% This is a LaTeX document. LaTeX is a markup language for producing documents.
% Your task is to answer the questions by filling out this document, then to
% compile this into a PDF document.
% You will then upload this PDF to `Gradescope' - the grading system that we will use.
% Instructions for upload will follow soon.
%
%
% TO COMPILE:
% > pdflatex thisfile.tex
%
% If you do not have LaTeX and need a LaTeX distribution:
% - Departmental machines have one installed.
% - Personal laptops (all common OS): http://www.latex-project.org/get/
%
% If you need help with LaTeX, come to office hours. Or, there is plenty of help online:
% https://en.wikibooks.org/wiki/LaTeX
%
% Good luck!
% James and the 1430 staff
%
%%%%%%%%%%%%%%%%%%%%%%%%%%%%%%%%%%%%%%%%%%%%%%%%%%%%%%%%%%%%%%%%%%%%%%%%%%%%%%%%%%%%%%%%%%%%%%%%
%
% How to include two graphics on the same line:
%
% \includegraphics[width=0.49\linewidth]{yourgraphic1.png}
% \includegraphics[width=0.49\linewidth]{yourgraphic2.png}
%
% How to include equations:
%
% \begin{equation}
% y = mx+c
% \end{equation}
%
%%%%%%%%%%%%%%%%%%%%%%%%%%%%%%%%%%%%%%%%%%%%%%%%%%%%%%%%%%%%%%%%%%%%%%%%%%%%%%%%%%%%%%%%%%%%%%%%

\documentclass[11pt]{article}
\usepackage{amsmath,amssymb,tikz}
\usetikzlibrary{arrows.meta,tikzmark}
\usepackage[english]{babel}
\usepackage[utf8]{inputenc}
\usepackage[colorlinks = true,
            linkcolor = blue,
            urlcolor  = blue]{hyperref}
\usepackage[a4paper,margin=1.5in]{geometry}
\usepackage{stackengine,graphicx}
\usepackage{fancyhdr}
\setlength{\headheight}{15pt}
\usepackage{microtype}
\usepackage{times}

\frenchspacing
\setlength{\parindent}{0cm} % Default is 15pt.
\setlength{\parskip}{0.3cm plus1mm minus1mm}

\pagestyle{fancy}
\fancyhf{}
\lhead{Project 1 Questions}
\rhead{CIE 552}
\rfoot{\thepage}

\date{}

\title{\vspace{-1cm}Project 1 Questions}


\begin{document}
\maketitle
\vspace{-3cm}
\thispagestyle{fancy}

\section*{Questions}

\paragraph{Q1:} Explicitly describe image convolution: the input, the transformation, and the output. Why is it useful for computer vision?

%%%%%%%%%%%%%%%%%%%%%%%%%%%%%%%%%%%
\paragraph{A1:} Convolution is the core operation in image and signal processing.It is mainly used to apply various filter types over images.Its importance comes from the fact that it has direct connection with the frequency domain of image ,as the convolution operation in the spacial domain can be converted into multiplication in the frequency domain and vice versa.The input of the convolution is two images with any size. the transformation itself occurs by the following equation.
\begin{equation}
(f\ast g)[n]=\sum^{\infty}_{m \rightarrow \infty} f[m]g[n-m]
\label{eq:one}
\end{equation}

The  output of  convolution is a single value for each iteration in the summation , so for an image with convolutional filter , each new pixel in the filtered image will be affected by the pixels next to it.It is also worth mentioning that the size of the filtered image is slightly less than the original one due to the pixels in the borders,hence we need padding to make the output filter with the same dimension as the original one.


%%%%%%%%%%%%%%%%%%%%%%%%%%%%%%%%%%%

% Please leave the pagebreak
\pagebreak
\paragraph{Q2:} What is the difference between convolution and correlation? Construct a scenario which produces a different output between both operations.

\emph{Please use \href{https://docs.scipy.org/doc/scipy/reference/generated/scipy.ndimage.convolve.html}{$scipy.ndimage.convolve$} and \href{https://docs.scipy.org/doc/scipy/reference/generated/scipy.ndimage.correlate.html}{$scipy.ndimage.correlate$} to experiment!}

%%%%%%%%%%%%%%%%%%%%%%%%%%%%%%%%%%%
\paragraph{A2:} They are basically the same operation ,but the convolution flips the filter firs before performing the multiplication and summation.in other words,correlation and convolution can be used interchangeably with taking into consideration the filter used.



%%%%%%%%%%%%%%%%%%%%%%%%%%%%%%%%%%%

% Please leave the pagebreak
\pagebreak
\paragraph{Q3:} What is the difference between a high pass filter and a low pass filter in how they are constructed, and what they do to the image? Please provide example kernels and output images.

%%%%%%%%%%%%%%%%%%%%%%%%%%%%%%%%%%%
\paragraph{A3:} Your answer here.



%%%%%%%%%%%%%%%%%%%%%%%%%%%%%%%%%%%

% Please leave the pagebreak
\pagebreak
\paragraph{Q4:} How does computation time vary with filter sizes from $3\times3$ to $15\times15$ (for all odd and square sizes), and with image sizes from 0.25~MPix to 8~MPix (choose your own intervals)? Measure both using \href{https://docs.scipy.org/doc/scipy/reference/generated/scipy.ndimage.convolve.html}{$scipy.ndimage.convolve$} or \href{https://docs.scipy.org/doc/scipy/reference/generated/scipy.ndimage.correlate.html}{$scipy.ndimage.correlate$} to produce a matrix of values. Use the \href{http://scikit-image.org/docs/dev/auto_examples/transform/plot_rescale.html}{$skimage.transform$} module to vary the size of an image. Use an appropriate charting function to plot your matrix of results, such as \href{https://matplotlib.org/tutorials/toolkits/mplot3d.html#scatter-plots}{$Axes3D.scatter$} or \href{https://matplotlib.org/tutorials/toolkits/mplot3d.html#surface-plots}{$Axes3D.plot\textrm{\_}surface$}.

Do the results match your expectation given the number of multiply and add operations in convolution?

\emph{Image:} \href{RISDance.jpg}{RISDance.jpg} (in the .tex directory).

%%%%%%%%%%%%%%%%%%%%%%%%%%%%%%%%%%%
\paragraph{A4:} Your answer here.



%%%%%%%%%%%%%%%%%%%%%%%%%%%%%%%%%%%


% If you really need extra space, uncomment here and use extra pages after the last question.
% Please refer here in your original answer. Thanks!
%\pagebreak
%\paragraph{AX.X Continued:} Your answer continued here.



\end{document}
