%%%%%%%%%%%%%%%%%%%%%%%%%%%%%%%%%%%%%%%%%%%%%%%%%%%%%%%%%%%%%%%%%%%%%
%
% CSCI 1430 Writeup Template
%
% This is a LaTeX document. LaTeX is a markup language for producing
% documents. Your task is to fill out this
% document, then to compile this into a PDF document.
% You will then upload this PDF to `Gradescope' - the grading system
% that we use.
%
%
% TO COMPILE:
% > pdflatex thisfile.tex
%
% For references to appear correctly instead of as '??', you must run
% pdflatex twice.
%
% If you do not have LaTeX and need a LaTeX distribution:
% - Departmental machines have one installed.
% - Personal laptops (all common OS): www.latex-project.org/get/
%
% If you need help with LaTeX, please come to office hours.
% Or, there is plenty of help online:
% https://en.wikibooks.org/wiki/LaTeX
%
% Good luck!
% James and the 1430 staff
%
%%%%%%%%%%%%%%%%%%%%%%%%%%%%%%%%%%%%%%%%%%%%%%%%%%%%%%%%%%%%%%%%%%%%%
%
% How to include two graphics on the same line:
%
% \includegraphics[\width=0.49\linewidth]{yourgraphic1.png}
% \includegraphics[\width=0.49\linewidth]{yourgraphic2.png}
%
% How to include equations:
%
% \begin{equation}
% y = mx+c
% \end{equation}
%
%%%%%%%%%%%%%%%%%%%%%%%%%%%%%%%%%%%%%%%%%%%%%%%%%%%%%%%%%%%%%%%%%%%%%%%%%%%%%%%%%%%%%%%%%%%%%%%%

\documentclass[11pt]{article}

\usepackage[english]{babel}
\usepackage[utf8]{inputenc}
\usepackage[colorlinks = true,
            linkcolor = blue,
            urlcolor  = blue]{hyperref}
\usepackage[a4paper,margin=1.5in]{geometry}
\usepackage{stackengine,graphicx}
\usepackage{fancyhdr}
\setlength{\headheight}{15pt}
\usepackage{microtype}
\usepackage{times}
\usepackage{booktabs}

\frenchspacing
\setlength{\parindent}{0cm} % Default is 15pt.
\setlength{\parskip}{0.3cm plus1mm minus1mm}

\pagestyle{fancy}
\fancyhf{}
\lhead{Image filtering project}
\rhead{CIE 552}
\rfoot{\thepage}

\date{}

\title{\vspace{-1cm}Hybrid image illusion using convolutional filtering}

\usepackage{listings}
\usepackage{color}

\definecolor{codegreen}{rgb}{0,0.6,0}
\definecolor{codegray}{rgb}{0.5,0.5,0.5}
\definecolor{codepurple}{rgb}{0.58,0,0.82}

\lstdefinestyle{mystyle}{
    commentstyle=\color{codegreen},
    keywordstyle=\color{magenta},
    numberstyle=\tiny\color{codegray},
    stringstyle=\color{codepurple},
    basicstyle=\footnotesize,
    breakatwhitespace=false,
    breaklines=true,
    captionpos=b,
    keepspaces=true,
    numbers=left,
    numbersep=5pt,
    showspaces=false,
    showstringspaces=false,
    showtabs=false,
    tabsize=2
}

\lstset{style=mystyle, language=python , frame=single}

\begin{document}

\maketitle	

\vspace{-3cm}
\thispagestyle{fancy}


\section*{Introduction}


Human eye's perception depends strongly on the distance between the object and the observer. If the distance was small, the eye start focusing more on the sharp details of the image (high frequency),however at the far away distances, the eye is only capable of detecting sooth variations (low frequency).In this project we are trying to use that phenomena to create an illusion by making hybrid image, which gets its low frequency content from an image and its high frequency content from another one.


\section*{Implementation of the convolutional filter}

This function is used to perform convolution operation over RGB and gray scale image . It has two types of padding (zeros and mirror).It takes only odd shaped kernel,and returns the output in the form of an image with the same size as the input image

\begin{lstlisting}
def my_imfilter(image, kernel,mode ='zeros'):
	# get the kernel dimesnsions in kh and kw
	kh,kw=kernel.shape
	# get the hight and width only as the third dimension may exist or not
	ih=image.shape[0]
	iw=image.shape[1]
	# get the number of dimsensions for both the kernel and the image
	kdim=kernel.ndim
	idim=image.ndim
	
	assert kdim == 2 , "kernel dimensions must be exaxctly two"
	assert idim == 2 or idim ==3 , "image dimensions must be exaxctly two or three"
	assert (kh%2) !=0 and (kw%2) !=0 , "all kernel dimensions must be odd"
	# this foarmula calculates how many rows do i need to put so that i can make proper padding
	hpad=(kh-1)//2
	wpad=(kw-1)//2
	
	filtered_image = np.zeros_like(image)
	
	if mode=='zeros':
		md='constant'
	elif mode=='reflect':
		md='reflect'
	else:
		raise   Exception('the mode {} is not defined \n "zeros" and "reflect are available"'.format(x))
	# change the padding function according to the input image
	if idim==2:
		paddedImg=np.pad(image,[(hpad,hpad),(wpad,wpad)],mode=md)
	else:
		paddedImg=np.pad(image,[(hpad,hpad),(wpad,wpad),(0,0)],mode='constant')
		dim_No=image.shape[2]
	# apply convolution over the number of channels
	for dim in range(0,dim_No):
		for i in range(0,ih):
			for j in range(0,iw):
				# multiply the kernel with the cropped part of the image and then sum the result
				
				cropped=paddedImg[i:i+kh,j:j+kw,dim]
				filtered_image[i,j,dim]=np.sum(np.multiply(kernel,cropped))
	
	return filtered_image

\end{lstlisting}

\section*{Filter tests and results}

{\Large \textbf{Identity filter}}

\begin{figure}[h]
    \centering
%    \includegraphics[width=5cm]{placeholder.jpg}
%    \includegraphics[width=5cm]{placeholder.jpg}
    \caption{\emph{Left:} My result was spectacular. \emph{Right:} Curious.}
    \label{fig:result1}
\end{figure}


\end{document}
